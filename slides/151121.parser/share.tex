\documentclass[c]{beamer}
\usepackage{CJK}
\usepackage{listings}
\usetheme{Madrid}
\renewcommand\mathfamilydefault{\rmdefault}
\usepackage{pgfplots}
\begin{document}
\begin{CJK}{UTF8}{song}

  \newcommand{\hm}{热力图}
  \newcommand{\hmsnd}{热力图二期}
  \newcommand{\hmfst}{热力图一期}
  \newcommand{\rerd}{实时渲染}
  \newcommand{\steam}{后台服务}
  \newcommand{\cteam}{客户端}
  \newcommand{\toi}{云控制}
%%------------------------------------------
\title[\hm{}串讲]{\hm{}串讲}
\author[lichao06]{lichao06}
\subtitle{\hmsnd{}\rtrd{}介绍}
\date{\today}
\frame{\titlepage}
\lstset{breakatwhitespace=true,
    language=Python,
    basicstyle=\footnotesize\ttfamily,
    keywordstyle=\color{red},
    frame=lrtb,
    numbers=right,
    extendedchars=true}
%%------------------------------------------

\begin{frame}\frametitle{\hm{}项目简介}
    \framesubtitle{\hm{}功能}
    \hm{}的主要功能:
    使用图片的方式,展示某一个时间段内,用户在空间上的分布情况
    \begin{figure}[htbp]
      \centering
      \includegraphics[scale=.2]{imgs/sample.png}
      \caption{\hm{}效果图}
      \label{fig:hmsample}
    \end{figure}
\end{frame}

\begin{frame}\framesubtitle{\hm{}原理}
  \framesubtitle{heatmap.js}
  热力图调研阶段,采用的是开源的「heatmap.js」作为绘图算法(https://github.com/pa7/heatmap.js)。

  基本算法如下:
    \begin{itemize}
        \item 根据配置生成调色板
        \item 申请目标区域需要占用的画布(w*h个像素点,每个点占用四个字节rgba)
        \item 在画布上,根据数据点的座标和数值,在中心点为圆心,R为半径的圆内,绘制alpha值逐渐变小的图像
        \item 如果画布上已经有以前绘制过的图像,将新、旧两个值进行「融合」
        \item 全部数据点绘制完成后,使用第一步生成的调色板,根据画布上每个点的alpha值进行着色
    \end{itemize}

    \pause
    用一张图来解释渲染过程:
    \begin{figure}[htbp]
      \centering
      \includegraphics[scale=.2]{imgs/render.png}
      \caption{\hm{}渲染过程}
      \label{fig:hmrender}
    \end{figure}
\end{frame}

\begin{frame}\frametitle{\hmfst{}的实现}
    \begin{figure}[htbp]
      \centering
      \includegraphics[scale=.2]{imgs/original.png}
      \caption{\hmfst{}渲染过程}
      \label{fig:hmrender}
    \end{figure}
\end{frame}

\begin{frame}\frametitle{\hmfst{}的实现}
    \framesubtitle{移植、抓取和预冲}

    热力图一期项目,在实现上重写了heatmap.js的实现,采用数据预冲的方式,在客户端请求之前,将所有的热力图渲染完成并写入缓存。
    \pause
    工作流程如下:
    \begin{itemize}
        \item 数据准备: 抓取定位组1/10线上服务器中,10min定位日志; 从中解析出用户位置数据,并写入mysql
        \item 图片渲染:使用步骤一生成的数据,根据预先设置好的城市列表(及其对应的范围),使用前面提到的渲染方式,渲染出一整个城市的热力图
        \item 图片预冲:图片渲染完成后,将整张画布切成瓦片图,转换为png之后写入redis等待客户端查询
        \item 切换版本:全部城市的渲染、预冲结束后,会切换系统热力图版本,此后客户端读到的全部是最新数据
    \end{itemize}
    \pause
    \hmfst{}存在的问题:
    \begin{itemize}
        \item 数据延迟过长: 经过一次改进后,数据准备、渲染和预冲需要花费大约30min的时间
        \item 支持比例尺较少: 需要预先申请画布,17级和更大的比例尺,会发生oom
    \end{itemize}
\end{frame}

\begin{frame}\frametitle{\hmsnd{}}
    \begin{figure}[htbp]
      \centering
      \includegraphics[scale=.2]{imgs/new.png}
      \caption{\hmsnd{}渲染过程}
      \label{fig:hmrender}
    \end{figure}
\end{frame}

\begin{frame}\frametitle{\hmsnd{}}
    \framesubtitle{\rtrd{}的实现}
    为了解决数据延迟、大比例尺下oom的问题,二期项目使用了实时渲染的问题:不需要预先渲染全部的热力图数据,只需要在客户端请求到来时,渲染对应区域的图片即可。

    它从以下两个方面解决了一期存在的两个问题:
    
    \begin{itemize}
      \item 延迟: 实时渲染并不需要预冲全部的图像数据,所以数据消耗基本等于下载、分析日志的时间
      \item oom: 由于不需要一次画完整个城市的图片,可以将内存消耗控制在可控范围内
    \end{itemize}

    \hmsnd{}引入的新问题
    \begin{itemize}
      \item 渲染、图片转换都是cpu intensive操作,耗时较长
      \item 由于热力图的原理,两次图片的边界,需要考虑拼接问题
    \end{itemize}
\end{frame}
\begin{frame}\frametitle{\hmsnd{}}
    \framesubtitle{\rtrd{}引入问题的解决}
    
    \begin{itemize}
      \item 减少数据量
        将全部数据,划分为边长为x的矩形,对落在全部矩形内的点计算重心,然后每个矩形只会产生一点数据。计算量随数据量减少,可以减小渲染时间

      \item 扩大画布+裁剪,解决图片无法拼接的问题
        根据前面热力图的工作方式,即「每个数据点只会影响自己周围半径R范围内的像素点」这一特性,可以计算出数据点影响到的最大半径,记为R'。
        每次渲染部分热力图的时候,会将目标范围的左上、右下扩大R', 渲染完成后,再将边界剔除,剩下的数据即为有效数据。oom: 由于不需要一次画完整个城市的图片,可以将内存消耗控制在可控范围内
    \end{itemize}
\end{frame}

%\begin{frame}[fragile]
%\frametitle{WHY}
%    \framesubtitle{逻辑与控制}
%    如何对一个列表a里的元素加1赋值给列表b
%    \pause
%\begin{lstlisting}[language=Python]
%for i in range(len(a)):
%    b[i] = a[i] + 1
%\end{lstlisting}
%\pause
%    如何对一个列表a里的元素平方后赋值给列表b
%    \pause
%\begin{lstlisting}[language=Python]
%for i in range(len(a)):
%    b[i] = a[i] * a[i]
%\end{lstlisting}
%\pause
%    如何对一个列表a里的元素\textcolor{gray}{加2}赋值给列表b \\
%    如何对一个列表a里的元素\textcolor{gray}{平方根后}赋值给列表b \\
%    ......
%
%\end{frame}

\end{CJK}
\end{document}

